\newpage
\TOCadd{ABSTRACT}

%\noindent \textbf{Supervisory Committee}
%\tpbreak
%\panel

%\begin{center}
{
	\centering
	\fontsize{17.5}{0}
	\textbf{ABSTRACT}
\par}
%\end{center}

In this thesis, we propose two new approaches which aim at improving robustness in social fabric-based cultural algorithms. Robustness is one of the most significant issues when designing evolutionary algorithms. These algorithms should be capable of adapting themselves to various search landscapes. \newline In the first proposed approach, we utilize the dynamics of social interactions in solving complex and multi-modal problems. In the literature of Cultural Algorithms, Social fabric has been suggested as a new method to use social phenomena to improve the search process of CAs. In this research, we introduce the Irregular Neighborhood Restructuring as a new adaptive method to allow individuals to rearrange their neighborhoods to avoid local optima or stagnation during the search process.\newline 
In the second approach, we apply the concept of Confidence Interval from Inferential Statistics to improve the performance of knowledge sources in the Belief Space. This approach aims at improving the robustness and accuracy of the normative knowledge source. It is supposed to be more stable against sudden changes in the values of incoming solutions.\newline 
The IEEE-CEC2015 benchmark optimization functions are used to evaluate our proposed methods against standard versions of CA and Social Fabric. IEEE-CEC2015 is a set of 15 multi-modal and hybrid functions which are used as a standard benchmark to evaluate optimization algorithms. We observed that both of the proposed approaches produce promising results on the majority of benchmark functions. Finally, we state that our proposed strategies enhance the robustness of the social fabric-based CAs against challenges such as multi-modality, copious local optima, and diverse landscapes.