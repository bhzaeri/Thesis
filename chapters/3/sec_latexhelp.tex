\section{Evolutionary Algorithms}
Evolutionary computing is a research area within computer science. As the name suggests, it is a particular flavor of computing, which draws inspiration from the process of natural evolution. In fact, they are computer programs which try to solve and optimize complex problems by simulating the behavior of natural systems. They utilize evolutionary operators such as crossover and mutation to improve the quality of a population of solutions \cite{burke2005search}. \newline
Evolutionary algorithms start with a population of randomly generated solutions for a particular problem. Then, the algorithm modifies the solutions in an iteration-based process of some finite number of generations. The modification occurs through applying the so-called evolutionary operators such as crossover and mutation. The evolutionary operators might be binary or unary. For example, the crossover is a binary operator because it combines two solutions (individuals) to generate a new one. On the other hand, mutation is a unary operator because it makes random modifications on a single solution to improving its performance \cite{eiben2003introduction}. The performance of each is evaluated using a fitness function. The fitness function represents the problem that needs to be solved or optimized. Usually, the fitness function is chosen from NP problems which are hard to solve by traditional problem-solving methods. In each iteration of an evolutionary algorithm, the evolutionary operators are applied to the individuals and the best performing offsprings will be transferred to the next generation. This process continues until meeting some stopping criteria which are already defined by a human user. Some examples of evolutionary lgorithms are:
\begin{enumerate}
	\item Genetic Algorithms
	\item Evolutionary Porgramming
	\item Differential Evolution
\end{enumerate}
However, the evolutionary algorithms are not restricted to biological processes. Some researchers have drawn inspiration from social systems to find solutions to complex problems. The complex and coordinated behavior of swarms not only fascinates biologists but also is an inspiration to computer scientists \cite{bonabeau1999swarm}. Ant colonies and birds flocking are remarkable examples of coordinated collective behavior that emerges without centralized control. Swarm intelligence is a field of computer science that invents computational methods for solving problems in a way that is inspired by the behavior of real swarms and colonies. Principles of self-organization and communication (local and indirect) are essential to understanding the complex collective behavior \cite{stutzle2009ant}. There are different types of swarm-based algorithms such as:
\begin{enumerate}
	\item Particle Swarm Optimization (PSO)
	\item Ant Colony Optimization (ACO)
	\item Cultural Algorithms (CA)
\end{enumerate}
\section{Cultural Algorithms}
Cultural Algorithms (CA) were introduced by Reynolds as a type of population-based problem-solving approaches. CA combines biological evolution with socio-cognitive concepts to yield an optimization approach based on a dual inheritance theory \cite{reynolds1994introduction}. As defined by \cite{durham1991coevolution}, culture is a \enquote{system of symbolically encoded conceptual phenomena that are socially and historically transmitted within and between populations}.From the definition, it can be stated that in cultural systems, evolution occurs at two levels: Macro-evolutionary level and micro-evolutionary level. Cultural algorithms define the evolution process through the cooperation of three distinct components: 
\begin{enumerate}
	\item Belief Space (Macro-evolutionary Level)
	\item Population Space (Micro-evolutionary Level)
	\item Communication Protocol
\end{enumerate}
%In the first component, a population-based algorithm, such as GA, PSO, and EP, is deployed. It is called micro-evolutionary level too. In the second component (belief space) or macro-evolutionary level, there are five types of knowledge sources.
\subsection{Belief Space}
Belief space keeps different kinds of knowledge obtained from the individuals' experience during the evolution process. It extracts the knowledge from the population and stores the knowledge in various formats called knowledge sources (KS). Each knowledge source extracts and keeps knowledge about a particular aspect of the search space. 
\newline
After each iteration, the best performing individuals are chosen and sent to the belief space to be utilized by the knowledge sources. The stored knowledge is used to bias the search process in the population space. Five types of knowledge sources have been identified in the belief space:
\begin{enumerate}
	\item Situational Knowledge: Successful exemplars of individuals
	\item Normative Knowledge: The best range of value for each parameter
	\item Topographic Knowledge: The best areas of the search space
	\item Domain knowledge: The domain ranges for all the parameters and the best exemplars
	\item Temporal Knowledge: Knowledge about the past changes in a dynamic environment
\end{enumerate}
\subsection{Population Space}
In the population space, any population-based algorithm could be used. In the earlier variants of cultural algorithms, only GA was used. However, researchers utilized other population-based algorithms such as PSO and EP in the population component. In each generation, the best individuals are sent to the belief space to update the knowledge sources. Then, the belief space influences the population through the search process. Figure \ref{fig:StandardCA} describes the components of cultural algorithms.
\subsection{Communication Protocol}
In cultural algorithms, the belief space and the population communicate together through a two-way protocol. The population space sends the best individuals to the belief space using the acceptance function to update the knowledge sources. Then, the belief space bias the search direction of the population space through the influence function.
\begin{figure}[h]
	\includegraphics[scale=0.09]{StandardCA}
	\centering
	\caption{Cultural Algorithms \cite{kobti2013heterogeneous}}
	\label{fig:StandardCA}
\end{figure}
\section{Social Fabric-based Cultural Algorithms}
Social Fabric is a dynamic grid of information flow in which the individuals' interactions happen. The fabrics (networks) are created by the connectivity of each agent with other agents in a dynamic structure. The topology of the network controls the rate and type of interactions \cite{reynolds2008social}. Like a multi-population model, there are multiple independent subpopulations (tribes) which are networked together \cite{ali2012socio}.\newline
In the Social Fabric idea, the influence of the belief space is propagated through the population using a multi-layered network of connections. There are Z tribes comprised of H individuals which form two layers: rudimentary and advanced \cite{ali2012socio} \cite{ali2016leveraged}. The members of each tribe are connected in a regular topology independent from other tribes. At this level, they form the rudimentary layer. From each tribe, the best performing individuals (elites) are connected. These connections create the advanced layer which is responsible for mediating the flow of information between different tribes.
\subsection{Evolution Phases}The whole evolution process is split into three phases: seclusion, rapport, and cohesive. In the first stage, each tribe evolves as an independent basic CA model with no communication between tribes. In the second step, the advanced layer is formed by connecting elites of each tribe. Then, tribes start to exchange information during their evolution process. Ultimately, in the third step, all the tribes are merged into one CA model. Then the search process continues until meeting some stopping criteria.
\subsection{Strategic Neighborhood Restructuring}In Social Fabric literature, Strategic Restructuring is a technique to help individuals to get rid of stagnation where there are copious local optima in the search landscape. Each tribe maintains a particular topology until it gets stagnated in a local optimum for a certain number of iterations. Then, the topology is reinitialized to motivate the individuals for a new search. Here, four types of topologies are utilized: Ring, Mesh, Hybrid-Tree, and Global \cite{ali2016leveraged}. Figure \ref{fig:StandardNR} describes how the process of neighborhood restructuring occurs.
\begin{figure}[h]
	\includegraphics[scale=0.4]{StandardNR}
	\centering
	\caption{Standard Neighborhood Restructuring Process}
	\label{fig:StandardNR}
\end{figure}
\subsection{Social fabric based influence function} In this model, the individuals might decide to change their controller KS using the majority of KSs which they receive from their neighborhood. Here, Majority Voting is used to find the controller KS of an individual \cite{che2010robust}. Figures \ref{fig:SFNetworkModel} describes the social fabric influence function. In the case of a tie, some tie-breaking rules are deployed such as MFU (most frequently used), LFU (least frequently used), Direct (the direct influencing KS), Random (a random choice), Last-used (the KS which controlled the KS in the last iteration) \cite{ali2012socio}. In the equation \ref{eq:4} the sum denotes the counts of KSs in $i$th node neighborhood and $\psi_{i}$ is its contoller KS. $\tau_{i}$ is the net affecting KS for the next iteration \cite{ali2016leveraged} \cite{sterling2004aggregation}. 
\begin{equation}
\label{eq:4}
\tau_{i}=\sum_{j \in Nbr(i)}m_{ij} + \psi_{i}
\end{equation}
The Social Fabric approach can be generalized on different topologies as follows:
\begin{equation}
	KS(t+1)=
		\begin{cases}
			KS_{i}, & \forall KS_{j}\in \{KS-KS_{i}\}\Rightarrow weight(KS_{i}) > weight(KS_{j}) \\
			KS_{cr}, & otherwise
		\end{cases}
\end{equation}
Where $KS$ is the set of all knowledge sources, $KS_{i}, KS_{j}\in KS$. $weight(ܵKS_{i})$ is the number of neighbors that belong to the knowledge source $KS_{i}$. $KS_{cr}$ denotes the knowledge source chosen by a tie-breaking rule.
\begin{figure}[h]
	\includegraphics[scale=0.4]{SFNetworkModel}
	\centering
	\caption{Social Fabric Influence Function\cite{5284844}}
	\label{fig:SFNetworkModel}
\end{figure}
\section{Particle Swarm Optimization}
The initial idea for particle swarm optimization of Kennedy and Eberhart were essentially aimed at producing computational intelligence by utilizing simple models of social interaction, rather than purely single-agent cognitive capabilities.\newline
In PSO, some simple-structured individuals (the particles) moving around in the search space of a function, and each particle evaluates the objective function based on its current location. Then, each particle determines its movement direction and velocity through the search space by combining its historical best experience and the best (best-fitness) particle in its visible neighborhood in the swarm, with some random fluctuations for keeping diversity. This process repeats at each iteration. Eventually, the swarm as a whole, like a flock of birds collectively searching for food sources, is similar to move around for finding an extremum of the fitness function.\newline
Each particle is comprised of a position vector and velocity vector \cite{bratton2007defining}. Particles adjust their velocities and positions as follows:
\begin{equation}
v\textsubscript{id}\textsuperscript{(t+1)}=w\cdot v\textsubscript{id}\textsuperscript{(t)} + c\textsubscript{1}r\textsubscript{1}(p\textsubscript{id}-x\textsubscript{id}\textsuperscript{(t)})+c\textsubscript{2}r\textsubscript{2}(p\textsubscript{gd}-x\textsubscript{id}\textsuperscript{(t)})
\end{equation}
\begin{equation}
x\textsubscript{id}\textsuperscript{(t+1)}=x\textsubscript{id}\textsuperscript{(t)}+v\textsubscript{id}\textsuperscript{(t+1)}
\end{equation}
Where $v\textsubscript{id}$ and $x\textsubscript{id}$ are the velocity and position of $i$th particle. $c1$ and $c2$ are two positive constants. $r1$ and $r2$ are randomly generated numbers in [0,1] range. $w$ is the inertia weight of which restricts the velocity of a particle. $p\textsubscript{id}$ is the best experience of the particle and $p\textsubscript{gd}$ is the best solution in its neighborhood. Figure \ref{fig:PSOModel} describes how the particle interact together in PSO.
\begin{figure}[h]
	\includegraphics[scale=0.4]{PSOModel}
	\centering
	\caption{Social interaction model of PSO}
	\label{fig:PSOModel}
\end{figure}
\subsection{Tribe-PSO}
Similar to Social Fabric model, the whole population is divided into some tribes.  The tribes form a two-layer structure of networks, and the whole process consists of three phases \cite{chen2006tribe}. The individuals in each tribe are completely isolated from other tribes. So, they form the basic layer. On the other hand, the best members of each tribe can see the other tribes and exchange information with their best members as well. The problem-solving process is divided into three phases: isolated, communing, and united. In the first phase, the tribes are completely isolated and there is no exchange of information. In the second phase, the two-layered model is formed and the tribes begin to exchange information through their best performing individuals. In the third phase, all the tribes are merged into a single population. Then, it operates as basic PSO model until meeting some stopping criteria. In fact, the idea of Social Fabric originates from Tribe-PSO. But the network structure is utilized to propagate the influence of knowledge sources in the belief space \cite{ali2016leveraged}. In the thesis, we use the social fabric for both PSO interactions and knowledge propagation of the belief space. Figure \ref{fig:TribePSO} describes the architecture of Tribe-PSO.
\begin{figure}[h]
	\includegraphics[scale=0.6]{TribePSO}
	\centering
	\caption{Tribal-PSO}
	\label{fig:TribePSO}
\end{figure}

\section{Chapter Conclusion}
This chapter presents a detailed description of evolutionary and social methods such as cultural algorithms, social fabrics, and PSO. The chapter was focused on Tribal variants of before mentioned approaches and how to deploy social relationships among the tribes' members. 







%
%
%
%A Latex document is composed of two parts: the Preamble, and the Document Body.  The \textit{Preamble} is the site for inclusion of all document set up commands: definition of new commands, inclusion of pre\-built packages, template declaration, etc.  The \textit{Body} is where the document content is placed.
%
%\subsection{Preamble}
% The Preamble refers to the input which precedes the documents contents.  It is the area where the author determines the general template for the document using the \textbf{$\backslash$documentclass}$[options]$\{\textit{doc style}\} command.  For example,
% $\backslash$documentclass$[11pt]$\{\textit{article}\} declares that a document will follow the \textit{article} document class, and have 11pt font.
% \newline\\
% If the document requires support of any library packages they must be included in the preamble using the \textbf{$\backslash$usepackage}\{\textit{package name}\} command. For example, $\backslash$usepackage\{graphicx\} is the command needed to include the graphicx package.
%
%\subsection{Document Body}
%The document body is the area which follows the Preamble. It is defined by the \textbf{$\backslash$begin}\{\textit{document}\} and \textbf{$\backslash$end}\{\textit{document}\} commands.  The content of a Latex document is declared in the document body.  Input which appears after the $\backslash$end\{document\} command is ignored.
%
%\section{How to Number Pages}
%To number the pages of a document use the \textbf{$\backslash$pagenumbering}\{\textit{style}\} command. Numbering is defined in the documents preamble. There are several different \textit{styles} to choose from.%\\\vspace{-2mm}
%\begin{table}[h]
%    \begin{center}
%        \begin{tabular}{|l|l|}
%            \hline 
%            \textrm{\textbf{Numbering Style}}   & \textrm{\textbf{Output}} \\ \hline
%            $\backslash$pagenumbering\{arabic\} & 1, 2, 3, ...\\ \hline
%            $\backslash$pagenumbering\{roman\}  & i, ii, iii, ...\\ \hline
%            $\backslash$pagenumbering\{alph\}   & a, b, c, ...\\ \hline
%            $\backslash$pagenumbering\{Roman\}  & I, II, III, ...\\ \hline
%            $\backslash$pagenumbering\{Alph\}   & A, B, C, ... \\ \hline
%        \end{tabular}
%        \caption{Page Numbering Styles}
%        \label{tb:Xname}
%    \end{center}
%\end{table}
%
% The numbering of pages for a thesis is, however, much more complex than for an article and, in fact, the \textit{book} class has been adopted. Make changes to those settings only if you are really familiar with \LaTeX.
%
% \section{How to Create a Title Page}
% A title page can be either on a separate page or integrated directly into the first page of the document. It is defined by three declarations, followed by the \textbf{$\backslash$maketitle} command as illustrated below.\\\vspace{-2mm}
% \begin{center}
%  \texttt{
%  \begin{tabular}{l}
%    $\backslash$title\{Title of Paper\}\\
%    $\backslash$author\{Author(s) of Paper\}\\
%    $\backslash$date\{Publication Date\}\\
%    $\backslash$maketitle
%  \end{tabular}
% }
% \end{center}
% The article document class defaults on an integrated title page.  To make a separate title page, use the \textbf{titlepage} option with the
% \mbox{$\backslash$documentclass$[titlepage]$\{\textit{doc style}\}} command.
%
% For this thesis style the title page has been completely formatted for you. Just insert the various names of people in the supervisory committee, the title, your name and so on in the location where the \textit{dummy} entries exist right now and you will be done. I would suggest to avoid doing any other changes unless you are absolutely sure!
%
%\section{How to Create an Abstract}
% To create an abstract, place contents of abstract between the \textbf{$\backslash$begin}\{\textit{abstract}\} and \textbf{$\backslash$end}\{\textit{abstract}\} commands.
%
%\section{How to Create a Table of Contents}
% The \textbf{$\backslash$tableofcontents} command automatically generates a table of contents from all section headers.  The default behavior for the
% article document class is to produce an integrated table of contents.  However, the document can be altered to generate the table of contents on a
% separate page using the \textbf{$\backslash$newpage} command (see section Formatting Extras).
%
% For this thesis template a special command has been added, namely the \textbf{$\backslash$textTOCadd}. You can find it
% in the file \textit{macros/style.tex}. It has to be explicitly called for an insertion into the Table of Contents and it is already in place appropriately for the existing sections and subsections.
%
%\section{How to Create Sections}
%Creating sections, subsections, and subsubsections is completed using the $\backslash$section\{Section Name\}, $\backslash$subsection\{Subsection Name\} and $\backslash$subsubsection\{Subsubsection Name\} commands, respectively.  Each sectional division is numerically labeled with respect to it's placement in the section hierarchy. For example, this section was defined with the code:
%
%\begin{center}
%  \texttt{
%  \begin{tabular}{l}
%    $\backslash$section\{How to Create Sections\}\\
%    Creating sections, subsections, and ...
%  \end{tabular}
% }
%\end{center}
%
%It is useful to give a label using the \textbf{$\backslash$label} command to a section or subsection if a reference to it is made, so that the reference will be automatically updated should the structure of the document change.
%
%\section{How to Create a List}
% Lists can be either enumerated, non enumerated, or descriptive. Each element of a list is termed an 'item'.
%
% \begin{enumerate}
%    \item enter the list environment with the $\backslash$begin\{\textit{list style}\} command.
%    \vspace{-2mm}
%    \item define each item with the $\backslash$item command for non$\backslash$enumerated lists, or $\backslash$item$[\textit{label}]$ for descriptive lists.
%    \vspace{-2mm}
%    \item terminate list environment with the $\backslash$end\{\textit{list style}\} command.
% \end{enumerate}
%
%\section{How to Insert Tables, Figures, Captions, and Footnotes}
% The table and figure environments contain input blocks which cannot be split across pages.  Rather than divide the input of either of these
% environments, the contents are relocated, or floated, to a location in the document which optimizes page layout with the surrounding document content.
%
%\subsection{Tables}
% Tables are created in the tabular environment. A single parameter is used to define the number of columns and item justification pertaining to each column.  The single parameter is a combination of the following ones shown in Table \ref{tb:example1}.
% \\
%\begin{table}
% \begin{center}
% \begin{tabular}{|l|l|} \hline
%    \textbf{loc}    &   \textbf{Purpose}        \\ \hline
%    l               &   left justified column   \\ \hline
%    r               &   right justified column  \\ \hline
%    c               &   centered column         \\ \hline
%    $|$             &   vertical rule           \\ \hline
% \end{tabular}
% \end{center}
% \caption{Table Example}
% \label{tb:example1}
%\end{table}
%$\backslash$$\backslash$ and \& are used to define rows and columns, respectively.  A table can either have the contents of its rows and columns lined or not.  Each line used to construct the table must be individually specified, using $|$ and $\backslash$hline for vertical and horizontal lines, respectively.
%
%Table \ref{tb:example1} was generated with the following input:
%
%\begin{verbatim}
%            \begin{center}
%                \begin{tabular}{|l|l|} \hline
%                l & left justified column   \\ \hline
%                r & right justified column  \\ \hline
%                c & centered column         \\ \hline
%                $|$ & vertical rule         \\ \hline
%             \end{tabular}
%             \end{center}
%\end{verbatim}
%
%You will want to include your table in the "List Of Tables"
%section at the beginning of your thesis. To do this you
%enclose the above table inside a table environment like so:
%
%\begin{verbatim}
%            \begin{table}
%                \begin{center}
%                ...
%                \end{center}
%                \caption{Sentence describing table.}
%                \label{unique:label}
%            \end{table}
%\end{verbatim}
%
%The caption is the text that appears underneath the table. 
%It should be short and precise. The label is a unique 
%label that you can use to refer to the table within 
%your document. You can use the \texttt{$\backslash$ref\{label\}}
%to insert the table number into your text as in Table \ref{tb:example1}.
%In the example above you would use as in:
%\begin{verbatim}
%            I am referring to Table \ref{unique:label}.
%\end{verbatim}
%
%\subsection{Figures}
% The first step to including an externally prepared image into a document, is to declare the graphixs package into the documents preamble.
% Integrating the image can be done using the figure environment. Enter and exit the figure environment with the \textbf{$\backslash$begin\{figure\}$[loc$]} and \textbf{$\backslash$end\{figure\}} commands, respectively. The \textit{loc} dictates the placement of the included image, and can be any of the following:
% \begin{description}
%    \item[h here:] location in text where the environment appears
%    \vspace{-2mm}
%    \item[t top:]  top of the page
%    \vspace{-2mm}
%    \item[b bottom:] bottom of the page
%    \vspace{-2mm}
%    \item[p page of floats:] on a separate page with no text
% \end{description}
%
% For organizational purposes, it is best to have keep all figures in a folder together. I usually label the folder as "\textit{Figures}" (with great creativity) and I placed it in the same directory as the topmost main \textit{.tex} file.  Include the image into the document
% with the \textbf{$\backslash$includegraphics}$[\textit{dim}]$\{path to image\} command.  \textit{dim} dictates the magnitude of the \texttt{\textbf{height}} or \texttt{\textbf{width}}.  The image is scaled proportionally.  An example and its resulting output follow below.
% \\    \vspace{-2mm}
% \begin{center}
% \texttt{
% %\begin{tabular}{l}
% $\backslash$begin\{figure\}$[h]$\\
%    \hspace{.25in} $\backslash$centering\\
%    \hspace{.25in} $\backslash$includegraphics$[height=1in]$\{LinuxPenguin.eps\}\\
%    \hspace{.25in} $\backslash$caption\{The Linux Penguin\}\\
% $\backslash$end\{figure\}
% %\end{tabular}
% }
% \end{center}
%
% % Use this figure inclusion with a file of type eps
% % An example is included
% % Compile (at least in MicTex) using the LATEX button
% % followed by the DVIPDF button
% % vspace{-2mm}
% %\begin{figure}[h]
% %   \centering
% %   \includegraphics[height=1in]{Figures/linux_small.eps}
% %   \caption{The Linux Penguin}
% %   \label{fig:penguineps}
% %\end{figure}
%
% % Use this figure inclusion with a file of type pdf, jpg
% % gif , etc. but not eps
% % An example is included
% % Compile (at least in MicTex) using the PDFLATEX button
% %\vspace{-2mm}
% %\begin{figure}[h]
% %   \centering
% %  \includegraphics[height=1in]{Figures/LinuxPenguin.pdf}
% %   \caption{The Linux Penguin}
% %   \label{fig:penguinpdf}
% %\end{figure}
%
%Why is the output for the figure not shown? Because inserting figures into \LaTeX is not that simple and it is highly dependent on the system you are using together with the type of figure. This is not the place to dwell upon the inconsistencies which can make your life difficult. Suffice it to say that the original \LaTeX and its tools was geared to accept \textit{.eps} files for figures and it still maintains that expectation if one compiles using a \textit{Latex to dvi to (pdf or ps)} series of commands. On the other hand, if one uses the \textit{Latex to pdf} direct path, then files of other types are perfectly fine (e.g. \textit{pdf, jpg, gif, etc.}).
%
%If you are interested, look at the actual file for this section namely "sec\_latexhelp.tex"
%and consider the set of lines commented out just above this paragraph. There are two examples of insertion of figures, the first with the
%\textit{.eps} version and the second with the \textit{.pdf} version of the same picture (of a penguin). Delete the comments from one of the two sets and use the appropriate tools.
%
%To refer to a figure, the same approach used for tables should be used, namely a \texttt{$\backslash$ref\{label\}} command 
%which includes the unique identifier label for that figure, as
%in:
%\begin{verbatim}
%            I am referring to Figure \ref{unique:label}.
%\end{verbatim}
%
%\subsection{Captions}
% Captions for tables and figures are created using the \textbf{$\backslash$caption}\{caption goes here\}. Captions are automatically numbered with separate counters for tables and figures. \textbf{$\backslash$caption}\{caption contents\} can only be used in the Figure or Table environment.
%
%\subsection{Footnotes}
% Footnotes are inserted with the \textbf{$\backslash$footnote}\{footnote contents\} command.  This footnote\footnote{this is a footnote} is generated as follows:
% \begin{center}
%    \texttt{...This footnote$\backslash$footnote\{this is a footnote\} is generated...}
% \end{center}
%
%\section{How to Alter Font}
%\subsection{Type Style}
% Roman Family is the default type style.  The types style can be modified using the following commands.
% \\\vspace{-2mm}
% \begin{center}
%    \begin{tabular}{ll}
%        \textbf{Command}                                & \textbf{Output} \\
%        $\backslash$textit\{Italic Characters\}         & \textit{Italic Characters}\\
%        $\backslash$textsl\{Slanted Chartacters\}       & \textsl{Slanted Characters}\\
%        $\backslash$textsc\{Small Cap Characters\}      & \textsc{Small Cap Characters}\\
%        $\backslash$textbf\{Boldface characters\}       & \textbf{Boldface characters}\\
%        $\backslash$textsf\{Sans Serif Characters\}     & \textsf{Sans Serif Characters}\\
%        $\backslash$texttt\{Typewriter Characters\}     & \texttt{Typewriter Characters}
%    \end{tabular}
% \end{center}
%
%\subsection{Type Size}
% The font size can be modified using the following commands.
% \\\vspace{-2mm}
% \texttt{
% \begin{center}
% \begin{tabular}{ll}
%    \textrm{\textbf{Command}}                   & \textrm{\textbf{Output}} \\
%    $\backslash$tiny\{tiny font\}               & \tiny{tiny font}\\
%    $\backslash$scriptsize\{scriptsize font\}   & \scriptsize{scriptsize font}\\
%    $\backslash$small\{small font\}             & \small{small font}\\
%    $\backslash$normalsize\{normalsize font\}   & \normalsize{normalsize font}\\
%    $\backslash$large\{large font\}             & \large{large font}\\
%    $\backslash$Large\{Large font\}             & \Large{Large font}\\
%    $\backslash$huge\{huge font\}               & \huge{huge font}\\
%    $\backslash$Huge\{Huge font\}               & \Huge{Huge font}
% \end{tabular}
% \end{center}
% }
%
%\section{Math Mode}
%To incorperate mathematical content into a document, Latex provides three different environments: Displaymath, Math, and Equation. Brief descriptions for each environment, and environment short cuts are displayed in the table below.
%\begin{center}
%\texttt{
%\begin{tabular}{||l|l|l||} \hline
%    \textrm{\textbf{Environment}} & \textrm{\textbf{Function}}           & \textrm{\textbf{Shortcut}} \\ \hline
%    math            & displays an in-text formula       & $\backslash$ $($ \ldots $\backslash$ $)$ \\ \hline
%    displaymath     & displays an unnumbered formula    & $\backslash$ $[$ \ldots $\backslash$ $]$ \\ \hline
%    equation        & displays a numbered formula       & N/A \\ \hline
%\end{tabular}
%}
%\end{center}
%
%The following examples, using Einstein's famous \( e \doteq mc^{2} \) equation, illustrate how to include a formula into a document.
%\\ \vspace{-5mm}
%\begin{verbatim} ...Einstein's famous \( e \doteq mc^{2} \) equation, illustrate... \end{verbatim}
% \begin{verbatim}
% \[e \doteq mc^{2}\]
% \end{verbatim}   \vspace{-17mm}  \[e \doteq mc^{2}\]
% \begin{verbatim}
% \begin{equation}
%    \doteq mc^{2}
% \end{equation}
% \end{verbatim}
% \vspace{-17mm}  \begin{equation}e \doteq mc^{2} \end{equation}
%\vspace{-2mm}
%
%\section{Formatting Extras}
%The following table illustrates some formatting tips for perfecting the layout of a Latex document.
%\begin{center}
%\texttt{
%\begin{tabular}{||l|l||}\hline
%        \textrm{\textbf{Command}}           & \textrm{\textbf{Purpose}}                      \\ \hline
%        $\backslash$hspace\{\emph{len}\}    & insert a horizontal space of length \emph{len} \\ \hline
%        $\backslash$vspace\{\emph{len}\}    & insert a vertical space of length \emph{len}  \\ \hline
%        $\backslash$mbox\{\emph{text}\}     & ensure that \emph{text} is not split over multiple lines   \\ \hline
%        $\backslash$$\backslash$            & new line \\ \hline
%        $\backslash$newpage                 & start new page  \\ \hline
%        $\backslash$pagebreak               & insert a page break \\ \hline
%        \%                                  & precedes comments \\ \hline
%\end{tabular}
%}
%\end{center}
%