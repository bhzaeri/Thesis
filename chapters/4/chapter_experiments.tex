\startchapter{Proposed Approaches: Improving Robustness in Social Fabric}
\label{chapter:Exp}
Inspired by natural systems, many evolutionary algorithms have been proposed to solve various optimization problems. These algorithms are expected to exhibit a consistent problem-solving behavior against different optimization landscapes. Unfortunately, as stated by No Free Lunch (NFL) theorem \cite{wolpert1997no}, there is no algorithm better than others over all cost functions. It means, there is no guarantee for an algorithm to work well for all functions if it shows promising results for a particular category of them. Therefore, robustness has been one of the most desired features which motivates researchers to invent new methods which are less dependent on the kind of a problem than others. Here, robustness means we need to develop search strategies which are capable of adapting themselves across different landscapes which might vary regarding aspects such as multimodality and the number of local optima. 
\newline
In this section, we are going to improve the robustness of CAs in both population space and belief space components. In the population space, a new neighborhood restructuring strategy will be proposed which aims at microscopic inspections for stagnation regarding individuals' local experience. Then, in the belief space, a new model of Normative KS will be proposed, which defines the normative ranges based on Confidence Interval inspired from Inferential Statistics. It stops the normative KS to be affected by sudden fluctuations in the input data.

\section{Self-organization}
A self-organized system is a group of agents with specific behavioral patterns which could not be predicted from the simple behaviors of the  individuals who make up the system. Kennedy and Eberhart \cite{kennedy2001swarm} described self-organization as follows: 
\begin{displayquote}
	"Self-organization The ability of some systems to generate their form without external pressures, either wholly or in part. It can be viewed as a system's constant attempts to organize itself into ever more complex structures, even in the face of the constant forces of dissolution described by the second law of thermodynamics."
\end{displayquote}
Some of the primary attributes of self-organizing systems are listed below:
\begin{enumerate}
	\item Self-organizing systems usually exhibit the behaviors that appear to be spontaneous order.
	\item Self-organization can be considered as a system's constant attempts to organize itself into a variety of complex structures, even in the face of the forces of deviation (Robustness).
	\item The overall status of a self-organizing system is an emergent property of all the ingredients of the system.
	\item Interconnected components of the system become organized in a meaningful way based on local interactions of the components.
	\item Complex systems have the potential to self-organize themselves.
	\item The self-organization process works close to the "edge of chaos".
\end{enumerate}
In the section, we will propose a new approach to improve the robustness of cultural algorithms regarding self-organization.
\section{Irregular Neighborhood Restructuring}
As proposed by \cite{ali2016leveraged}, when the agents of a tribe get stuck in a local optimum, then they choose a topology with fewer connections such as ring topology to motivate exploration. When the agents lack information, the topology is changed to a denser topology like Mesh, and ultimately Global. Whenever a tribe does not make progress for $M_{thresh}$ iterations, then the algorithm decides on upgrading/downgrading the current topology to other topologies. Downgrading happens by moving in the direction of gbest$\rightarrow$tree$\rightarrow$mesh$\rightarrow$lbest and Upgrading occurs in the direction of lbest$\rightarrow$mesh$\rightarrow$tree $\rightarrow$gbest. In our proposed approach, the restructuring process occurs at a microscopic level. There is no daemon process for inspecting stagnation. Every agent checks its performance for stagnation. If it gets stuck in a local optimum, it decides to upgrade/downgrade its neighborhood. The proposed strategy is described in the algorithm \ref{ref:INR}. If the particle is the best in its neighborhood, it starts to decrease the number of connections to motivate exploration. However, if it is not the best, it needs to increase the neighborhood size to facilitate exploitation. Here, the topology is dynamic and irregular and the neighborhood size is different for each agent. So, it could be considered as a heterogeneous variant of CAs \cite{de2009heterogeneous}. The ultimate topology of the fabric is formed through local interactions of the agents. In this way, it improves the robustness of CAs regarding self-organization concept \cite{kennedy2001swarm}. Psudo-code \ref{ref:INR} describes the irregular neighborhood restructuring algorithm in detail.
\begin{algorithm}{	}
	\caption{Irregular Neighborhood Restructuring}			
	\eIf{$fitness(x\textsubscript{i}) > bestSoFar\textsubscript{i} $}{
		\eIf{$stagnationCounter \ge TiggerThreshold$}{
			\uIf{$x\textsubscript{i} = x\textsubscript{best}$}{
				index = selected randomly from $x\textsubscript{i}$'s neighborhood\newline
				remove $x\textsubscript{index}$ from $x\textsubscript{i}$'s neighborhood
			}
			\Else{
				index = selected from $\{S - x\textsubscript{i}'s neighborhood\}$\newline
				add $x\textsubscript{index}$ to $x\textsubscript{i}$'s neighborhood
			}
			$stagnationCounter = 0$
		}{
		$stagnationCounter = stagnationCounter+1$
	}
}{
$stagnationCounter = 0$
}
\label{ref:INR}
\end{algorithm}
\section{Confidence based Normative Knowledge Source}
The current version of Normative KS is quite vulnerable to temporary fluctuations in the pattern of input data. The size of Normative ranges is subject to change dramatically with any temporal fluctuations. In this section, we are going to replace the ranges in the standard Normative KS with Confidence Interval from Inferential Statistics. Confidence-based changes are more robust against instantaneous changes in the input pattern. The update process of standard normative ranges is as follows:
\begin{equation}
lb_{j}(t+1) = \begin{cases} x_{i,j}(t), & x_{i,j}(t)\leq lb_{j}(t)\:or\:f(X_{i})<PL_{j}(t)  \\ lb_{j}, & \mbox{otherwise} \end{cases}
\end{equation}

\begin{equation}
ub_{j}(t+1) = \begin{cases} x_{i,j}(t), & x_{i,j}(t)\ge ub_{j}(t)\:or\:f(X_{i})<PU_{j}(t)  \\ ub_{j}, & \mbox{otherwise} \end{cases} 
\end{equation}%\newline
The following is the definition of normative ranges based on Confidence Interval concept \cite{proakis1985probability}.
%\begin{equation}
%	(\bar{x}_{j}-q\cdot\dfrac{\sigma_{j}}{\sqrt{n}} , \bar{x}_{j}+q\cdot\dfrac{\sigma_{j}}{\sqrt{n}}) 
%\end{equation}
\begin{equation}
lb_{j}=\bar{x}_{j}-q\cdot\dfrac{\sigma_{j}}{\sqrt{n}}
\end{equation}
\begin{equation}
ub_{j}=\bar{x}_{j}+q\cdot\dfrac{\sigma_{j}}{\sqrt{n}}
\end{equation}
where 
$\bar{x}_{j}$ is the mean of $j$th dimension, $\sigma_{j}$ is the standard deviation of $j$th dimension and $q$ is the confidence coefficient.




%Assuming you have some experimental results to support your claims this is where all the data is reported. There are a few issues you should consider before dumping a lot of stuff here, or it will lose its effectiveness.
%
%First of all you must describe precisely the experimental setup and the benchmarks you used. In any scientific discipline an experimental result is only good if it is reproducible. To be reproducible then somebody else must have sufficient details of the setup to be able to obtain the same data. Thus the first section in this chapter is a super precise history of the decisions made towards experimentation, including mentions of the paths which became infeasible. The setup must be valid and thus your description of it must prove that it is indeed sound. At times, terrifying times, when writing this section, both supervisor and student realize belatedly that something is missing and more work needs to be done!
%
%The second portion of this chapter is dedicated the the actual results. At least two issues arise here:
%\begin{enumerate}
%\item {Should all the data be reported here or should some be placed in the Appendix?}
%\item {Should this be an exposition of the raw facts and data or should it include its analysis and evaluation?}
%\end{enumerate}
%
%There are no definite answers here, but I follow a few rules.
%
%\textit{Should all the data be reported here or should some be placed in the Appendix?}
%    \begin{itemize}
%    \item{If there is a large number of tables of data, it might be better to present here only a handful of the most significant ("best") results, leaving all the rest of the data in the Appendix with proper linkages, as it would make the chapter so much more easily readable (not to mention limiting the struggle with a word processor for the proper placement of tables and text).}
%    \item{Use an example throughout, call it a "case study" to make it sound better, so that all the data and results are somehow linked in their logic, and even better if this is one of the examples you used in Chapter 2 to describe the original problem.}
%    \item{Highlight in some manner the important new data, for example the column of your execution speed where all the numbers are much smaller. Make the results highly easy to read!}
%    \item{It is normally expected that data should be presented only in one form and not duplicated, that is, you are not supposed to include both a table of raw numbers and also its graphical representation from some wonderful Excel wizard. I tend to disagree. I would not wish to see every results repeated in this manner, but some crucial ones need to be seen in different manners, even with the same information content, in order to show their impact. One good trick is to place the more boring tables in the Appendix and use wonderful graphs in this chapter.}
%    \item{This is the one chapter where I would splurge and use colour printing where necessary, as it makes an \textit{enourmous} difference.}
%    \end{itemize}
%
%\textit{Should this be an exposition of the raw facts and data or should it include its analysis and evaluation?}
%     \begin{itemize}
%    \item{Is the evaluation of the data really obvious? For example you have 10 tables to show that your chemical process is faster in development and gives purer material - you may simply need to highlight one column in each table and state the obvious.}
%    \item{Most results are not that obvious even if they appear so. Moreover this is where you are comparing your \textit{new} results to data from other people. I usually describe other people's work at this point and make comparisons. That is why I prefer to talk about the analysis and evaluation of the results in a separate chapter.}
%    \item{There is absolutely no clear structure here which is best.}
%    \end{itemize}
%