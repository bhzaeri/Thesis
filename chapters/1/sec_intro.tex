\section{Evolutionary Algorithms}
Evolutionary algorithms take their roots from the evolution theory of Darwin. These algorithms try to mimic the collective behavior of natural systems. Natural processes such as natural selection, survival of the fittest, and reproduction have been subject to inspiration as the fundamental components of evolutionary problem-solving methods. The basic part of all these algorithms is that they start with a randomly generated set of solutions. Then, they try to evolve the solutions through applying a set evolutionary operators such as mutation and crossover. \newline
Evolutionary algorithms are not limited to biological processes. There is another category of evolutionary algorithms known as Swarm Intelligence (SI). These algorithms take inspiration from social behaviors of living colonies such as ants, flocks, schools, and hives. Within these swarms, individuals have relatively simple structures, but their collective behavior usually looks very complex. The complex behavior of a swarm emerges as a result of the interactions between the individuals over time. This complex behavior can not be easily predicted by observing the simple behaviors of the agents separately. \newline
There are some well-known examples categorized as evolutionary algorithms:
\begin{enumerate}
	\item Genetic Algorithms
	\item Cultural Algorithms
	\item Prticle Swarm Optimization
	\item Ant Colony Optimization
	\item Honey Bee Colony
\end{enumerate}
Both categories of evolutionary algorithms share the same idea of evolving some initially generated solutions. However, the difference is in the way that they manipulate and evolve individuals through applying evolutionary operators. 
%\section{Robustness in Evolutionary Algorithms}
\section{Research Motivation}
The aim of this research work is to improve robustness in evolutionary algorithms.  In this field, robustness means that an algorithm can be used to solve many kinds of problems, with a minimum number of adjustments to address particular problems with special qualities. Also, it can mean the capability of algorithms to deal acceptably with noisy or missing data.\newline
As stated by No Free Lunch(NFL) theorem, there is no algorithm better than others over all cost functions. It means, there is no guarantee for an algorithm to work well for all functions if it shows promising results for a particular category of them. Therefore, robustness has been one of the most desired features which motivates researchers to invent new methods which are less dependent on the kind of a problem than others. In our thesis, we mostly emphaize on exploring different approaches to improve robustness in cultural algorithms regarding both belief and population spaces.\newline
\section{Thesis Contribution}
In the thesis, I am going to improve the robustness of Cultural Algorithms in both components of population and belief spaces. In the population space, a new neighborhood restructuring strategy is proposed which works based on a dynamic and irregular topology. In our approach, neighborhood restructuring occurs at a microscopic level, and every individual decides to change its neighborhood size. In the belief space, a new normative knowledge source is proposed which works based on the Confidence Interval concept inspired from Inferential Statistics. Also, we recognized that both Social fabric and PSO algorithm utilize the same social structure to facilitate social interactions between the individuals. So, we decided to use PSO in the population component of our proposed approach. We hypothesize that both of our approaches help the algorithm to resist against perturbations and not get affected by the fluctuations in the search space. The first approach lets the individuals adjust their relationships autonomously (self-organization). The second method makes the normative ranges in the belief space robust against fluctuations in the input data. \newline Various benchmark functions have been used to evaluate the efficiency of evolutionary algorithms. As a reliable benchmark, we chose IEEE-CEC2015 function set which covers most of the desired properties for an optimization testbed such as multi-modality, copious local optima, and non-separability. In our thesis, they are categorized in four categories: Unimodal, Simple multimodal, Hybrid, and Composite functions.
\section{Thesis Outline}
abcd