	\section{Evolutionary Algorithms}
Evolutionary algorithms take their roots from the evolution theory of Darwin. These algorithms try to mimic the collective behavior of natural systems. Natural processes such as natural selection, survival of the fittest, and reproduction have been subject to inspiration as the fundamental components of evolutionary problem-solving methods. The basic part of all these algorithms is that they start with a randomly generated set of solutions. Then, they try to evolve the solutions through applying a set evolutionary operators such as mutation and crossover. \newline
Evolutionary algorithms are not limited to biological processes. There is another category of evolutionary algorithms known as Swarm Intelligence (SI). These algorithms take inspiration from social behaviors of living colonies such as ants, flocks, schools, and hives. Within these swarms, individuals have relatively simple structures, but their collective behavior usually looks very complex. The complex behavior of a swarm emerges as a result of the interactions between the individuals over time. This complex behavior can not be easily predicted by observing the simple behaviors of the agents separately. \newline
There are some well-known examples categorized as evolutionary algorithms:
\begin{enumerate}
	\item Genetic Algorithms
	\item Cultural Algorithms
	\item Prticle Swarm Optimization
	\item Ant Colony Optimization
	\item Honey Bee Colony
\end{enumerate}
Both categories of evolutionary algorithms share the same idea of evolving some initially generated solutions. However, the difference is in the way that they manipulate and evolve individuals through applying evolutionary operators. 
\section{Robustness in Evolutionary Algorithms}