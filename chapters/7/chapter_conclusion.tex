\startchapter{Conclusion and Future Work}
\label{concl}
\section{Conclusion}
In this thesis, two novel strategies were introduced to improve the robustness of CAs in both population and belief spaces. In the first strategy, a new neighborhood restructuring strategy is employed which aims at individuals with irregular and heterogeneous neighborhood structures. There is no centralized coordinator to control the topology of a population because the individuals are responsible for inspecting and modifying their neighborhood in a self-organized manner. Increasing the level of self-organization and autonomy is a key approach to improving robustness in a complex system.  In the second strategy, the standard implementation of normative ranges in the belief space was replaced by the confidence intervals inspired from Inferential Statistics. The new knowledge source shows a more robust search behavior to fluctuations in the input data. Now, the size of normative ranges does not change dramatically with any temporal fluctuations.\newline
For the new neighborhood restructuring, a fundamental assumption is that the individuals have to handle their interactions within a social network (fabric) infrastructure. Here, the social fabric is modeled using a dynamic graph which the neighborhood size of the nodes (individuals) changes with time. This assumption implies another assumption that the interactions between agents occur locally. It means that the amount of influence which the individuals can receive from the other members is limited by their neighborhood size. In fact, this assumption is a key to avoiding immature convergence. As the agents are responsible for inspecting and modifying their neighborhood, the only limitation which we surmise when the size of a population increase, is the performance issues. Since the population is divided into some independent subpopulations (tribes), each tribe could be run in a separate thread. Running all the population in a single thread may pose a heavy burden on performance.\newline
The second approach, confidence-based normative knowledge source, does not assume anything more than its standard version. It is entirely independent of the strategies deployed in the population space. In fact, the confidence-based approach could be utilized in the same framework as well as the standard method. So, it does not pose any additional limitations to the algorithm.

\section{Future Work and Applications}
The performance of both proposed approaches is assessed through a test-suite of 15 multi-modal and hybrid functions from IEEE-CEC2015. Both methods are compared to the original version of the social fabric based CA \citet{ali2016leveraged}. The results show improvement in most of the functions. In some of them, the results are quite promising. \newline
We only tested our approaches against static functions, which means there is no change in the feedback from the environment. The improved robustness of the new methods makes them suitable to be utilized in dynamic problems like real-time applications where the algorithm needs to inspect and adapt itself to a changing environment continuously and as fast as possible \cite{che2010robust}. The new neighborhood restructuring method allows the individuals to develop relationships in promising parts of the search space autonomously. The deployed inspect and adapt strategy makes it easy for the agents to change their point of focus and restructure themselves to find new optimum points in a non-stable search space. Also, the new update scheme for the normative ranges in the belief space can be considered suitable for such applications since it avoids deviation with any temporal change in the input data pattern.\newline
With the growing industrial demands for data analytics, evolutionary algorithms could be the source of inspiration for the solutions to address such problems. The self-organization aspect of EAs makes them scalable to be used for analyzing datasets with a high level of complexity. \cite{xiaodong2008web}, \cite{pizzuti2008ga}, and \cite{sherkat2015structural} are the examples which show promising results in social network analysis. They used EA methods such as GA, CA, and PSO for community detection and link prediction applications. \cite{sousa2004particle} and \cite{otero2012inducing} are the examples which proposed new approaches for data classification based on PSO and ACO. So, these research works have created a good background to use our proposed strategies for analytic applications. 

%My first rule for this chapter is to avoid finishing it with a section talking about future work. It may seem logical, yet it also appears to give a list of all items which remain undone! It is not the best way psychologically.

%This chapter should contain a mirror of the introduction, where a summary of the \textit{extraordinary} new results and their wonderful attributes should be stated first, followed by an executive summary of how this new solution was arrived at. Consider the practical fact that this chapter will be read quickly at the beginning of a review (thus it needs to provide a strong impact) and then again in depth at the very end, perhaps a few days after the details of the previous 3 chapters have been somehow forgotten. Reinforcement of the positive is the key strategy here, without of course blowing hot air.

%One other consideration is that some people like to join the chapter containing the analysis with the only with conclusions. This can indeed work very well in certain topics.

%Finally, the conclusions do not appear only in this chapter. This sample mini thesis lacks a feature which I regard as absolutely necessary, namely a short paragraph at the end of each chapter giving a brief summary of what was presented together with a one sentence preview as to what might expect the connection to be with the next chapter(s). You are writing a story, the \textit{story of your wonderful research work}. A story needs a line connecting all its parts and you are responsible for these linkages.
