\startchapter{Related Works}
\label{chapter:problem}

\newlength{\savedunitlength}
\setlength{\unitlength}{2em}

Many variants of CAs have been proposed in a vast range of different applications such as single and multi-objective optimization, dynamic problems, social interactions simulation. Here, we are interested in studying socially motivated and multi-population variants as some modern approaches in solving optimization problems.
\subsection{Heterogeneous Multi-Population Cultural Algorithm}
[] proposed a new architecture for cultural algorithms. In this pproach, the whole population is divided into a set of independent sub-populations which work in parallel without direct communication. They referred to the works of [] [] [] []. As their motivation they stated that most of proposed variants of evolutionary algorithms suffers from immature convergence. This occurs because these algorithms can not hold the diversity at a reasonable level. Based on existing research works, they hypothesized that multi-population streategies would be a better choice as they have the potential to perfrom an efficient search on complex landscapes. In their approach, the optimization parameters are divided among some heterogenous sub-populations. the sub-populations are called heterogenous because each sub-population is responsible for optimizing a different subset of parameters. Each sub-population represents a partial solution instead of a complete solution. To evaluate a partial solution, it gets completed by its complement parameter values from the belief space. The complete solution is evaluated based on a numerical optimization function. Also, to make the convergence process faster a simple local search strategy is incorporated into the proppsed algorithm. The general architecture of their algorithm is presented in figure ???\newline
In the experiments, they considered the whole population size to be 1000 individuals. It is divided into 30 sub-populations. So, the size of each sub-popualtion is 33. The algorithm runs for the maximum of 10000 generations and the local search strategy runs only for 10 iterations. They evaluated HMP-CA on a set of 8 complex optimization functions. It is able to find the minimum value for 7 functions out of 8. However, when the local search strategy is applied to the expriments, the proposed method outperforms all of the functions and it finds the optimum value very quickly. Ultimately, they claimed that their porpsed approach is efficient in both time and space complexity.

\subsection{The Social Fabric Approach as an Approach to Knowledge Integration in Cultural Algorithms}	
[] begins with a brief introduction to socially motivated methods to problem solving. It compares qualities of Ant Colony Optimization (ACO), Particle Swarm Optimization (PSO), and Cultural Algorithms (CA) regarding the scale in which the interactions between agents occur. Figure (()) compares PSO, ACO, and CA in terms of the time and space continuum over which the social interactions occur. Individuals in ACO and PSO tend to interact in a reatively limited temporal and spatial scales. It is obvious because the agents in both ant and paricle swarm algorithms exchange information with only other agents in their local neighborhood. On the other hand, cultural algorithms let the individuals interact together using various types of symbolic information emerged from complex cultural systems. In cultural algorithms the interactions among individuals occur indirectly through a shared belief space. So, cultural algorithms allow individuals interact in a global scale.\newline
Then, they asked the essential question of what social structures might emerege alongisde the search process?. To answer such questions, they introduced a new influence function which utilizes the social fabric phenomena. The old influence function assumes no interactions between agents and works based on the simple roulete wheel method. On the other hand, in the new influence function, the individuals are connected through a social network (fabric). Multiple layers of such networks could be employed in a population. The interplay of these network connection forms a social fabric. At each iteration, an individual could specify its controller knowledge source. In this approach, the contoller knowledge source is chosen based on the majority of knowledge sources in the neighborhood of an individual. The neighborhood size of an individual is specified by the topology of the fabric. Inspired by Particle Swarm Optimization literature, different topologies could be taken into consideration to model the relationships among individuals. In their work, they only considered Ring and Square topologies. They stated that, the topology of the social fabric determines the extent to which the influence of knowledge sources could be spread thorugh the network.\newline
To evaluate the social fabric approach, they implemented it in Repast frameork. Repast is a simuation tool for multi-agent systems. They created a cultual algorithms toolkit (CAT) to view the capabilities of cultural algorithms in solving various problems. They chose Cone World problem to evaluate and compare their approach with the standard cultural algorithm. The reason that they chose this problem is that by changing its parameters during the evolution process, it can show a dynamic behavior. So, Cone World problem provides an efficient way to test flexibility of search algorithms. They set the parameters of CAT as: 100 individuals, 100 cones and 1000 generations. They used ring and square topologies to from he social fabric. They stated that square topology works better than ring as it finds the solution after 250 iterations. While, the ring topology finds the best solution 450 iterations.
\subsection{Robust Evolution Optimization at the Edge of Chaos: Commercialization of Culture Algorithms}
The authors of [] aimed at commercializing Cultural Algorithm Toolkit (CAT) thourgh developing a robust variant of it. By robustness, they mean to develop a cultural algorithm which is capable of being applied across a vast range of complex problems. At first, they referred to [Peng model] as an standard model cultural algorithms which assumes no connection between individuals. Then, they referred to [Ali], which introduced the concept of social fabric to allow individuals interact together. The authors extended the work of Ali by allowing the social networks having a memory. In addition, they utilized a variety of different networks in order to deteremine the relationship between network and problem complexity. \newline They brought up the hypothesis that there might be multiple independent networks for different purposes such as kinship and economics. In such networks, there are always some individuals which are member of multiple networks and so, they can play the role of mediator between differet networks. To preserve the diversity, the authors utilized a variety of different topologies such as Lbest, Square, Hexagon, Octagon, Hexadecagon, and Gbest. \newline
They stated that in the previous work of [Ali], he employed an un-weighted majority win strategy to determine the contoller knowledge source of an individual. It just relied on the count of each knowledge source in the neighborhood of each individual. The authors replaced it with a new strategy which use the average fitness of each KSs instead of each KSs count. So, the performance of individuals in a neighborhood influences which knowledge source to be chosen for the next iteration. \newline
To evaluate the robustness of the algorithm, the authors used Cone World Generator [] as a dynamic problem. Reffered to the work by [Langton] ,they defined three classes of entropy in the connes world problem:
\begin{enumerate}
	\item Fixed: There is a low entropy and the parameters of the environment do not tend to change at all.
	\item Periodic: The environment parameters change in a regular period of time. So, the algorithm should be capable adapting itself regularly.
	\item Chaotic: In this case, the parameters change without any order and no prediction could be made about them. In fact, it is more similar to real-world cases.
\end{enumerate}
For the fixed category, the square topology successfully solved 84\% of the problems and as the best topology. For the periodic case, the octagon topology showed a better performance than other toplogies. And, for the chaotic category, Gbest showed a better result mostly in terms of average time to find a solution and the standard deviation. In summary, as the complexity of problems grows, there is the need for more connections between individuals in a social fabric to keep the robustness of the algorithm.
\subsection{Socio-Cultural Evolution via Neighborhood-Restructuring in Intricate Multi-Layered Networks}
The authors of [] aimed at exploring the utilization of neighborhoods in the population level of cultural algorithms and see how it influences the knowledge swarming in the belief space. Their approach uses a dynamic neighborhood restructuring to preserve diversity efficiently. They referred to the reasearch works of [], [], []. Those papers proposed some adaptive methods which tried to make a trade-off between exploration and eploitation properties of evolutionary algoithms. Among the referred papers, the work by [ALi] used the social fabric phenomena to from multi-layered hierarchical network structures in both homogenous and heterogeneous networks.\newline
The authors defined the social fabric phenomena as follows:
\begin{displayquote}
	"The Social Fabric is a living informational skin created out of the engineered emergence of agents illustrating the tension between the individual and the community in a context of interaction between them"
\end{displayquote}
The informational skin is created by the connectivity of individuals together. The social fabric is used to combine the behaviors at both indivials and society levels. Then, they proposed a strategy to determine the contoller knowledge source of each individual in each generation. In the strategy, the agents send the name of their contoller knowledge source to their neighbord through the social fabric. Each individual picks the mostly used knowledge source in its neighborhood as its controller for the next iteration. In case of a tie between two or more knowledge sources, three tie-breaking strategies are intorduced:
\begin{enumerate}
	\item Most Frequently Used (MFU)
	\item Random
	\item Least Frequently Used (LFU)
\end{enumerate}
They stated that their proposed approach is inspired from a previous work of Ali[22]. The new idea was called Cultual Algorithm with Restructuring Layered Social Fabric (CARLSOF). In this approach, the whole population is divided into two layers of rudimentary and advanced members. There are some independent tribes in the population space which their best performing individuals form the advanced layer and the others form the rudimantary level. In the previous works, the topology of social fabrics were supposed to be fixed and homogenous. They utilized a variety of regular topologies to form the social fabric. The authors proposed a new strategy to enable the fabric to be restructured in the case of stagnation. Each tribe may decide to change its topology to a denser (with more conections) or sparser (with less connections). They referred to two different restructuring strategies as similar approaches. The first one was Layered Delaunay Triangulation which is based on voronoi diagram. The second approach was Random Rewiring Procedure which starts with a regular topology and then rewire each edge randomly with the probability $p$. \newline
To evaluate the performance of CARLSOF, they used function set of IEEE-CEC2011 evolutiosry competition. The algorithm was implemented in JAVA. The authors reported that CARLSOF was successful to enhance all of the functions in the testbed interms of average, best, and worst obtained values. Also, they claimed that CARLSOF outperforms the best previously obtained results for European Space Agency (P12) and Casini (P13) problems results. They reported 2.983 km/sec compared to previous best 7.095. For p13, they reported 8.383091 km/sec compared to previous best 8.3832.
\subsection{Tribe-PSO: A novel global optimization algorithm and its application in molecular docking}
The authors proposed a new variant of Particle Swarm Optimization (PSO) called Tribe-PSO for the primary purpose of molecular docking in chemometrics. As some previous research work, they referred to [], [], [], []. Their approach is inspired from Hierarchical Fair Competition concept []. They divided the whole population into some tribes and the evolution process into three phases. The tribes are organized into two layers of basic and upper individuals. \newline The individuals in each tribe are completely isolated from other tribes. So, they form the basic layer. On the other hand, the best members of each tribe can see the other tribes and exchange information with their best members as well. The problem-solving process is divided into three phases: isolated, communing, and united. In the first phase, the tribes are completely isolated and there is no exchange of information. In the second phase, the two-layered model is formed and the tribes begin to exchange information through their best performing individuals. In the third phase, all the tribes are merged into a single popualtion. Then, it operates as basic PSO model until meeting some stopping criteria. The autohrs claimed that their approach helps the individuals preserving diversity and avoiding local optima against multi-modal solution spaces.\newline
The authors used two testbeds to evaluate and compare the performance of Tribe-PSO with standard PSO. The first was De Jong's function set and the second was a test set of 100 receptor-ligand X-ray structures selected for docking benchmark. In the De Jong's testbed, the basic PSO showed a better performance in the isolated phase. However, in the communing phase Tribe-PSO converged to a better value than basic PSO. In the unity phase, the basic PSO seemed to get stuck in a local optimum. However, Tribe-PSO continued to accelerate convergence and got better results than basic PSO. In the docking benchmark, Tribe-PSO was compared to the AutoDock library. Four parameters are calculated after 10 independent runs: Best, Run1, Average, and Standard Deviation of the results. The relative difference of the four benchmark factors between Tribe-PSO and AutoDock were calculated. The results for the four factors showed that Tribe-PSO leads to a better performance than AutoDock.
\subsection{Heterogeneous Particle Swarm Optimizers}
The authors peresented a survey on heterogeneous variants of Paritcle Swarm Optimization (PSO). They claim that the homogenous models have been attractive because of their simplicity in conceptual and application levels. However, heterogeneous models are ubiquitous in nature. Here, heterogeneity means the individuals may differ from each other regarding their parameters and search behavior. \newline 
At first, the authors presented a breif description of three well-known variants of PSO: Accelerated PSO[ref], Fully-informed PSO[ref], and Bare Bones PSO[ref]. Then, they authors categirzed the heterogeneous variants of PSO into four categories:
\begin{enumerate}
	\item Neighborhood
	\item Model of Influence
	\item Update Rule
	\item Parameters
\end{enumerate}
Neighborhood heterogeneity refers to the cases that the topology of the swarm is not regular. This type of heterogeneity occurs when the neighborhood size of each individual is different. They claimed that the neighborhood heterogeneity allows some population to be more influential than otehrs. Individuals with higher number of connections have the potential to attract more individuals through the search process. [Kennedy ref]\newline
Model of Influence heterogeneity refers to the situations that the individuals employ different strategies to specify their informers. The word Informer refers to an individual which is going to influence another individual. []\newline
Update-rule heterogeneity means the individuals utilize different strategies to update their position in the search space. This kind of heteogeneity makes it possible for the individuals to explore the search space in several ways. Also, the particles can play different but complementary roles. For example, some of them may tend to explore unseen parts of the solution space and some others only follow those scout individuals. The second type are the exploiters.\newline
Parameter heterogeneity occurs when some individuals in a group which follow the same update rule use different update rule's parameter settings. Having different search parameters, even in a group of similar particles, leads to various search behaviors which improves the level diversity. The authors referred to [], [] ,[] which utilize different initial values for the parameters such as acceleration coefficient, maximum velocities and inertia weight.\newline
To compare the mentioned categories, the authors careted two test cases. The first one compared two PSO variants with different update rules: Velocity-based and Bare bones swarm. The evaluation results confirmed that velocity-based outperformed the other one. The second test case compared two PSO variants with different models of influence: best-of-neighborhood and fully-informed swarm. The evaluation results showed that the fully-informed algorithm outperfromed the best-of-neighborhood.
\subsection{Leveraged Neighborhood Restructuring in Cultural Algorithms for Solving Real-World Numerical Optimization Problems}


\newline------------------------------\newline

\newline

\newline

Here is where you tell me what is the problem you have been working on for the past few months (or years). I want all the details and you should not be timid about being too tutorial, except that you do not want to cross the line towards writing a textbook. However consider carefully that \textit{communication} implies conveying ideas to other people, while \textit{effective communication} occurs when your message is clearly understood. Remember that your audience must understand your message before they can agree with you.

Ask yourself:
\textit{who is your audience?} You might think of your supervisor who knows everything and you want to impress with your knowledge. I think instead of the graduate students who will be reading this thesis which is, after all, a property of the university. It is published as a university technical report so that others may learn by reading it. Then teach them! Be a bit tutorial. Even the expert external examiner will be impressed by your clarity of exposition if he or she does not need to read paragraphs twice in order to understand - something which people with PhDs and big egos find particularly irritating.

On the other hand, do not go too far and give trivial definitions from concepts learned in a 3rd year undergraduate courses, else you might find yourself in trouble when having to remember the details during an oral examination.

My approach is to put everything necessary to make clarity for
the problem the main goal of this chapter, assuming an intelligent and well prepared reader who already has a Bachelor degree in an appropriate subject.

Once I understand the problem clearly and its nuances (it may not be what I expected after all), I also need to know why the problem is important, what its impact is and what its application, if any. Here you are free to elaborate and write as much as you think is necessary to avoid the examination doubt that you have a brilliant new solution to a trivial and unimportant issue.

I suggest reading various books on how to do research and set up problems. The best for me was "The Craft of Research" by Wayne Booth \cite{booth1}, which can be found in the main library at Q180.55 M4B66. From there I have transferred to my writing a fairly simple structure for talking about the topic of the research, with the question to be asked and its motivation and significance. It goes as follows:
\begin{enumerate}
\item {\textit{I am trying to learn about (working on, studying...)}}
\item {\textit{because I want to find out....}}
\item {\textit{in order to understand...}}
\end{enumerate}

Another way of looking at this is to ask the
\textit{what}, \textit{why} and \textit{where}, starting from a \textit{setting} of the problem with a first point A, stating what the \textit{goal} is at point B and having an \textit{action link} between the two which will encompass your new solution. As surprising as this may be to some of you, I found reading a book from Microsoft very useful: "Beyond Bullet Points: Using Microsoft Office PowerPoint 2007 to Create Presentations That Inform" \cite {atkin}. The goal of the book is to improve presentations with Power Point, but there is a lot that can be transferred towards \textit{effective communication} for a thesis.

In summary, my view of the second chapter on
\textit{"The Problem to be solved"} is as follows:
\begin{enumerate}
\item {\textit{Not} all the background and definitions (boring!) - use instead just-in-time explanations as needed in every context as it comes up;}
\item {Motivation in depth;}
\item {Tutorial high level explanation, where it is important to choose the right pitch: who is the audience? who are you teaching here?}
\item {Make it exciting, make it current, make it important - why do I want to keep reading?}
\item {Should you list here the solutions from other researchers? I think not, list instead the different facets of the problems that other researchers have attacked.}
\item {A taxonomy can be extremely useful to place your problem and its particular special features within the perfect context of the overall area, as you need to make sure that the reader understands perfectly what you are trying to solve.}
\end{enumerate}


\setlength{\unitlength}{\savedunitlength}
